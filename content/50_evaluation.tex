\chapter{Evaluation}
\label{sec:evaluation}

% Zu jeder Arbeit in unserem Bereich gehört eine Leistungsbewertung. Aus
% diesem Kapitel sollte hervorgehen, welche Methoden angewandt worden,
% die Leistungsfähigkeit zu bewerten und welche Ergabnisse dabei erzielt
% wurden. Wichtig ist es, dem Leser nicht nur ein paar Zahlen
% hinzustellen, sondern auch eine Diskussion der Ergebnisse
% vorzunehmen. Es wird empfohlen zunächst die eigenen Erwartungen
% bezüglich der Ergebnisse zu erläutern und anschließend eventuell
% festgestellte Abweichungen zu erklären.

The evaluation of the fast call mechanism is split into two parts: 
(i) we present the benchmarks for the existing kernel interface(system 
call, ioctl, and vDSO), our fast call, and another fast call mechanism that 
goes through the kernel.  (ii) we evaluate the impact of CPU side-channel attacks\cite{3,4} 
and propose our mitigation.

\section{Performance evaluation}

All benchmarks are based on  Intel Core i5-8400 processor 
with six cores at 2.8 GHz 6 hyper-threads(1 per core). 
The PC has 16 GB memory and runs Ubuntu 20.4 with the Linux 
kernel version 5.12. We disable the turbo boost and fix the 
CPU frequency at 2.8 GHz to get stable benchmarks. 
In addition, we also disabled address space randomization. 
Before running the benchmark, we use \emph{taskset} to set CPU 
affinity so that our benchmark can run in a specific CPU.  

We evaluate the performance of different IO methods, e.g., 
\emph{system call}, \emph{vDSO},  \emph{ioctil}, and fast calls created by two 
different fast call mechanisms. Note that fast calls created 
by another fast call mechanism go through the kernel compared 
to our mechanics.  The figure shows the five different IO 
methods we are going to compare. Technically, we have implemented a 
noop function for each method. In the benchmark, we use \emph{RDTSCP}\cite{14} to 
measure each method's time and repeat the measurement 1000000 times. 
In particular, we combine instruction \emph{CPUID} and \emph{RDTSCP} to serialize 
the out-of-order executed code.  After the measurement, we also 
calculate the mean, standard deviation, and median of each methods' 
execution time. The results are shown in the following figures.


\section{CPU side-channel attacks mitigation}

CPU side-channel attacks\cite{3,4} severely impact our fast call 
mechanism since fast call's components run on the user space, 
i.e., without kernel protection. Specifically, a fast call keeps its secret 
in the secret region, and the secret region relies on the permission bit on 
the page table entry to avoid any malicious user from reading this region. 
Remember that we discussed in the implementation chapter that the x86 architecture does not 
support executable-only pages.  Here, even if the x86 architecture supports 
only executable pages in the future, i.e., add a new permission bit for 
executable-only on \emph{PTE}\cite{25},  we cannot prevent attackers from using meltdown to 
attack this area. Now let us review the code snippet in the chapter \ref{sec:state} and see how an attacker uses this code to steal the secret(64-bit) in the 
secret region. Here we assume that the x86 architecture already supports 
executable-only pages.

\begin{lstlisting}[style=CStyle]
    char buf[8192]
  
    // the Flush of Flush+Reload
    clflush buf[0]
    clflush buf[4096]
  
    <some expensive instruction like divide>
  
    r1 = <a secret region virtual address>
    r2 = *r1
    r2 = r2 & 1      // speculated
    r2 = r2 * 4096   // speculated
    r3 = buf[r2]     // speculated
  
    <handle the page fault from "r2 = *r1">
  
    // the Reload of Flush+Reload
    a = rdtsc
    r0 = buf[0]
    b = rdtsc
    r1 = buf[4096]
    c = rdtsc
    if b-a < c-b:
      low bit was probably a 1
  \end{lstlisting}
  In this code snippet\cite{1}, the attacker first flushes the cache so that no 
  records related to \emph{buf[0]} and \emph{buf[4096]} exist in the cache. Then the 
  attacker lets the \emph{multi-core CPU} execute a costly instruction.  
  Instead of waiting for the completion of the instruction, the CPU 
  executes the following instructions speculatively. Remember, we discussed 
  in the chapter technical background that the CPU executes instructions 
  from lines 10 to 13 before checking whether the instructions are valid. 
  This means that the CPU checks the permission bit "executable only" on  
  \emph{PTE} corresponding to the page mapped to the secret region after it 
  obtains the secret stored on the secret region and complete the 
  code in line 12 and 13.  In other words,  in line 10 of the above 
  code snippet,  the CPU first tries to load the secret from the 
  cache directly. If it is a load miss, the secret data is fetched 
  from memory to register \emph{r2,} and the CPU puts it into the cache. 
  Then the CPU executes the instructions in lines 11, 12, 13 and 
  puts the value of \emph{buf[0]} or \emph{buf[4096]} to the cache based on 
  the first bit of the secret.  Later, during the instruction 
  retirement stage, the CPU checks whether the instruction in 
  line 10 is valid based on the permission bit on \emph{PTE} and reverts 
  its state since the execution from lines 10 to 13 are not permitted. 
  However, everything is too late. The value of \emph{buf[0]} or \emph{buf[4096]} 
  is already cached, and the CPU cannot revert the cache state. 
  Therefore, the attacker can then guess the secret by cache timing\cite{11}. 
  In the end, the attacker can get the secret by executing this code 
  snippet 64 times, each time guessing 1 bit of the secret.

  On top of that, the same things happened if we also set cache disable bit 
  on the secret page's \emph{PTE}.  In this case, any data 
  that is on the secret page will not be cached(the code in line 
  13 will not be cached).  This means,  in line 10 of the above code snippet, 
  after the secret is fetched from memory to register r2, the CPU won't 
  put it into the cache.  However, the meltdown attack does not rely on the 
  secret directly.  Instead, the attacker uses flush and reload to check 
  whether \emph{buf[0]} or \emph{buf[4096]} is in the cache. Therefore, it does not 
  matter whether the secret is in the cache.  In particular, there is no 
  way to avoid the buffer from being uncached since the attacker allocates 
  the buffer and does not set the \emph{cache-disable bit} on \emph{PTE}. Thus, we can't
   protect the secret region using the permission bit \emph{executable-only}
   and the cache-disable bit on \emph{PTE}.
  


  Because the meltdown attack is mainly composed of out-of-order 
  execution and covert channel, mitigation methods can start from 
  these two aspects. The second aspect would be more complicated 
  since there are multiple potential convert channels, including 
  cache, heat, etc.  So we decided to start from the first aspect. 
  Here our goal is to prevent the secret from being accessed by 
  unauthorized users. To achieve this goal, we argue that the CUP 
  must be redesigned.  More specifically, the CPU should execute 
  the memory-access related instruction(load data from memory or 
  store data to memory) in the following order\cite{1}:


  \begin{itemize}
    \item Before the speculative load operation, the CPU should first check the permission bits on PTU. 
    \item If the permission check is failed, zero should be returned instead of the real data.
  \end{itemize}
  In this way, the instructions in line 10 load zero into \emph{r2}. 
  Therefore, the attacker cannot guess the secret anymore. Particularly, the secret is not cached too since the CPU cache the secret after 
  it checks the permission bit. Note that AMD CPU 
  obviously works in this way all along, which means Meltdown does work on AMD 
  CPU.
 \cleardoublepage

%%% Local Variables:
%%% TeX-master: "diplom"
%%% End:
