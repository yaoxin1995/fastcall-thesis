% -*- Mode: Latex -*-

%  Zusammenfassung

% Zu einer runden Arbeit gehört auch eine Zusammenfassung, die
% eigenständig einen kurzen Abriß der Arbeit gibt. Eine halbe bis ganze
% DINA4 Seite ist angemessen. Dafür läßt sich keine Gebrauchsanweisung
% geben (für irgendetwas müssen die Betreuer ja auch noch da
% sein).

%\ldots abstract \ldots

%\todo{write abstract}

With the popularity of intelligent hardware like Smart NIC 
and NVMe based SSD, the transmission speed of hardware is getting faster and faster. 
This has led to that current I/O models are not suitable for data transmission due 
to Kennel's complex structure. When the application tries to communicate with 
the device through system calls, the kernel IO path's overhead becomes more and 
more severe in the context of NVMe-based SSD or RDMA network card. Thus, OS bypass 
technologies like DPDK and SPDK are coming to use to gain better performance 
and higher throughput. However, Those technologies are not flexible enough 
as they require dedicated hardware support, which is expensive and hard to adapt.

Clever developers may consider bypassing the kernel by directly 
mapping the device into userspace to work around the burden of 
specialized hardware. However, this has led to significant potential 
security issues because \textquote{ there is no centralized and trusted 
entity to control the operations issued by applications to hardware, 
and most devices themselves lack sufficient isolation mechanism}

We explore using a new system call mechanism to reduce the overhead
 caused by kernel and restriction of specialized hardware by mapping 
 system call functions and corresponding device control registers to 
 userspace. The evaluation shows this approach could significantly 
 increase the performance compared to the traditional IO path crossing 
 the user kernel boundary. However, we must avoid losing safety protection 
 for devices or mitigate secure vulnerability caused by bypassing the kernel
  entirely. We will try to sketch Potential solutions to these problems, 
  inspired by multiple classic attack models such as meltdown. 
  In summary, with this new idea, performance-critical applications 
  may have the ability to interact with PCIe devices in low latency,
   software-based, less secured way.
%%% Local Variables:
%%% TeX-master: "diplom"
%%% End:


