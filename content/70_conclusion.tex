\chapter{Conclusion and Future Work}
\label{sec:conclusion}

%  Schlußfolgerungen, Fragen, Ausblicke

% Dieses Kapitel ist sicherlich das am Schwierigsten zu schreibende. Es
% dient einer gerafften Zusammenfassung dessen, was man gelernt hat. Es
% ist möglicherweise gespickt von Rückwärtsverweisen in den Text, um dem
% faulen aber interessierten Leser (der Regelfall) doch noch einmal die
% Chance zu geben, sich etwas fundierter weiterzubilden. Manche guten
% Arbeiten werfen mehr Probleme auf als sie lösen. Dies darf man ruhig
% zugeben und diskutieren. Man kann gegebenenfalls auch schreiben, was
% man in dieser Sache noch zu tun gedenkt oder den Nachfolgern ein paar
% Tips geben. Aber man sollte nicht um jeden Preis Fragen, die gar nicht
% da sind, mit Gewalt aufbringen und dem Leser suggerieren, wie
% weitsichtig man doch ist. Dieses Kapitel muß kurz sein, damit es
% gelesen wird.


This paper introduces the motivation, design, implementation, 
and evaluation of the fast call mechanism for Linux kernel x86 
architecture, providing the user-level application with a 
high-performance IO path. The security issues that arise 
during the work of the fast call are taken into account. 
With fast calls, it is possible to have a high bandwidth 
communication between the application and device without 
exposing devices to unauthorized access.

The fast system call mechanism increases the IO performance using 
Kernel bypass, which means the user space application can communicate 
with devices through the fast path on the user space. Compared to 
traditional kernel bypass technologies, it also provides limited 
protection for devices directly mapped to the user space. 
Specifically, it leverages the permission bit on PTE and 
the address randomization technology to isolate the 
application and devices. On top of that,  fast calls 
are adaptable. In the application's point of view, 
invoking fast calls is the same as calling a user-level 
function, i.e., using function pointers.  However,  
fast calls are less secure than traditional approaches, i.e., 
system calls, because fast calls are running on the user space, 
losing the kernel protection.  This has also led to the fast 
call mechanism being particularly fragile in the CPU side-channel 
attacks. Therefore, we have proposed several ways to mitigate them 
in the evaluation chapter. 
We have compared the performance between the fast call and 
existing kernel interfaces, i.g. \emph{system call}, \emph{vDSO}\cite{9}, \emph{ioctl}, etc. 
The result shows that the fast call can improve the communication 
speed by over 20x.


Next, we plan to continue our work in the following three aspects:

\begin{itemize}
    \item We plan to generalize the fast call mechanism to more hardware architecture. 
    \item The fast call mechanism currently only supports the fast call service function written by the assembly. In the future, we plan to add fast call functions using the c language.
    \item We plan to migrate our implementation to Linux kernel version 5.14+ and employ the new Linux feature 
    \emph{memfd}\cite{27} to protect the fast call secret region further. The reason is that \emph{memfd} can avoid an attacker from exploiting the kernel's direct map to get the secret on the secret page.
\end{itemize}



\section{Acknowledgments}
Finally, I want to thank the supervisory team. Special 
thanks to M.Sc. Maksym Planeta and Dipl.-Inf. Till Miemietz for their suggestion, 
which helped me a lot in my thesis.





\cleardoublepage

%%% Local Variables:
%%% TeX-master: "diplom"
%%% End:
